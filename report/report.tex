\documentclass[12pt]{article}
\usepackage{geometry}
\geometry{
    a4paper,
    total={170mm,257mm},
    left=20mm,
    top=20mm,
}
\usepackage{titling}
\renewcommand\maketitlehooka{\null\mbox{}\vfill}
\renewcommand\maketitlehookd{\vfill\null}
\title{
    \Large Assignment 3 $\cdot$ CS32305 \\
    \Large Introduction to Computer Networks
    \rule[8pt]{\textwidth}{0.3pt}
    \LARGE Simulation of the TCP \\ Congestion Control Algorithm 
    \rule{\textwidth}{0.3pt}
}
\Large\author{Abhijit Hota $\cdot$ CE19B033}


\begin{document}
\maketitle

\newpage

\section*{Introduction}

The report should explain how different factors influence the $CW$ change over the duration of the session.

\subsection*{Notations \& Assumptions}

We repeat the notation and assumptions in the problem statement for the sake of completeness. We also add some of our own notations to ease the prose.

\vspace*{8pt}

\noindent\begin{tabular}{|c|p{0.66\textwidth}|c|}
    \hline
    \textbf{Notation} & \textbf{Description}                                                                           & \textbf{Range}            \\
    \hline
    $CW$              & Congestion window                                                                              &                           \\
    \hline
    $K_i$             & Initial CW                                                                                     & 1.0 $\le$ $K_i$ $\le$ 4.0 \\
    \hline
    $K_m$             & Multiplier of CW during the exponential growth phase                                           & 0.5 $\le$ $K_m$ $\le$ 2.0 \\
    \hline
    $K_n$             & Multiplier of CW during the linear growth phase                                                & 0.5 $\le$ $K_n$ $\le$ 2.0 \\
    \hline
    $K_f$             & Multiplier when a timeout occurs                                                               & 0.1 $\le$ $K_f$ $\le$ 0.5 \\
    \hline
    $P_s$             & Probability of receiving the \texttt{ACK} packet for a given segment before its timeout occurs & 0.0 $\le$ $P_s$ $\le$ 1.0 \\
    \hline
    $T$               & Number of segments to be sent for the emulation                                                &                           \\
    \hline
\end{tabular}
\subsection*{Methodology}

The task was to assess the congestion control algorithm of TCP with various parameters. A total of $6$ changeable parameters can be given to the simulation which affect the congestion window size.

\noindent
The following arguments were given to us.

\vspace*{8pt}

\noindent
$K_i \in \{1, 4\}, K_m \in \{1, 1.5\}, K_n \in \{0.5, 1\}, K_f \in \{0.1, 0.3\}, P_s \in \{0.99, 0.9999\}$
\vspace*{8pt}

\noindent
All combination makes a total of $32$ possible graphs. A value of $1000$ was taken for $T$. A Python script was created to iterate over all possible values and output the files to the path \texttt{/figures/*.svg} and \texttt{/figures/*.log}.

\vspace*{8pt}
\noindent
The Python script can be found in the code repository with the name \\ \texttt{simulate\_from\_parameters.py}

\newpage

\setcounter{page}{24}

\section*{Conclusion}

\begin{list}{$\bullet$}{}
    \item The $CW$ values depend on all of the parameters but they are also heavily dependent on the probability of successful \texttt{ACK}.
    \item The highest $CW$ matters in determining the shape of the graph.
    \item A higher $K_f$, $K_m$ and $K_n$ value generally give higher $CW$ values.
    \item An extra parameter can also be tested, i.e, the factor by which the congestion threshold was decreased.
    \item The formulae suggest an AIMD (Additive Increase Multiplicative Decrease) algorithm.
    \item The terms \emph{exponential} and \emph{linear} are kind of misnomers as during the phase where $CW$ is less than the threshold, it increases rapidly but linearly whereas during the later phase when the $CW$ is higher than the threshold, it increases in a logarithmic manner.
\end{list}

\end{document}